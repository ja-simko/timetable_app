%!TEX ROOT=ctutest.tex

\chapter{Úvod}

\section{Popis problému}
Nalezení nejkratší, potažmo nejrychlejší, cesty v síti je jedním z typických problémů, který se vyskytuje na dopravních sítích. Ať už se jedná o ranní dojížďku do zaměstnání vlastním automobilem nebo o lodní přepravu počítačových čipů napříč kontinenty, délka jízdní doby je jedním z klíčových faktorů, který se snaží jednotlivec nebo firma optimalizovat. Úspěšná optimalizace vede k ušetřeným finančním prostředkům a času, a je tedy cestou k nižším nákladům a efektivnějšímu řízení omezených zdrojů. 

V sítích veřejné dopravy hraje optimalizace tras spojů neopomenutelnou roli. Na jedné straně je cestující, pro kterého je očekávaná délka cestovní doby při volbě dopravního prostředku nebo trasy zásadní rozhodovací faktor. Dalším zásadním faktorem je cena dopravy. Vedlejší faktory jako přímé spojení, pohodlí nebo bezpečnost také ovlivňují volby cestujícího, avšak v menší míře. Na druhé straně je dopravce, jehož úloha je značně komplikovanější, jelikož krátká jízdní doba nezaručuje optimální řešení. Kratší jízdní doba může učinit řidiče efektivnější a snížit počet potřebných vozidel pro uspokojení poptávky, ale také může způsobit větší opotřebení vozového parku nebo drahé investice do moderních vozidel. 

Tato práce se však soustředí na ten první problém, tedy problém nalezení nejrychlejší cesty v rámci sítě veřejné hromadné dopravy (VHD). Problém nejkratší cesty se vyskytuje v mnoha odvětvích, a to také svědčí o spoustě algoritmů, které se používají pro každý specifický problém. Společné ale mají to, že se fyzická síť převede na abstraktní graf s vrcholy spojené hranami, nad nimiž algoritmus provádí operace.

Příklady, v nichž jsou algoritmy využity, je např. efektivní řízení toku elektřiny po elektrické rozvodné síti [\^11], pro nalezení spojení dvou lidí v sociálních sítích [\^12] nebo pro nalezení cesty pro charaktery v počítačových hrách. [\^13] Největší využití však algoritmy mají na dopravních sítích, ať už v navigacích řidičů osobní a nákladní přepravy, nebo diktující pohyb robotů v logistických centrech,  či jako pomocník vyhledávače spojení pro cestující v městských megalopolích. 

Mezi nalezením nejrychlejší cesty na silniční síti pro automobil a nalezením optimálního spojení ve veřejné hromadné dopravě je kritický rozdíl. Oba problémy mohou být převedeny do diskrétního grafu, avšak tradiční silniční síť je časově nezávislá, pokud není brána v potaz intenzita dopravy, a lze ji tak převést do statického grafu, kde každá hrana má fixní ohodnocení. Naopak spoje ve veřejné dopravě se řídí specifickým jízdním řádem, takže den a čas ovlivňuje ohodnocení hran; takovýto graf je dynamický. A právě vytvoření a implementace algoritmu nejkratší cesty v diskrétním dynamickém grafu je cílem této bakalářské práce.