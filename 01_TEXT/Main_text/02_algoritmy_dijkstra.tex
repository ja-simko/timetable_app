\chapter{Algoritmy nalezení nejkratší cesty}
\section{Dijkstrův algoritmus}

TEST \cite{RezvanianSocNetworks,SunPowerGrid} \cite{heimerl1988german} \cite{bauer2011experimental,lauter05Arcs} \cite{hilger2008fast} \cite{mohring2005partitioning} \cite{SchultesRoutePlanning} \cite{lazarsfeldGuide} \cite{ericksonShortestPaths}\cite{madkour2017survey} \cite{dibbelt2017CSA} \cite{tejaDynamic} \cite{CMUBinHeaps} \cite{pyrga2008efficient,PYRGA2004TowardsRealistic}

\cite{BatzTDCH,NanniciniBiDiA} \cite{}

Tento algoritmus pro nalezení nejkratší cesty byl poprvé popsán v odborném článku z roku 1959 nizozemským informatikem Edsgerem. W. Dijkstrou. \cite{Dijkstra59}[\^9] Dijkstrův algoritmus slouží pro nalezení nejkratší cesty z výchozího vrcholu do všech ostatních vrcholů v hranově ohodnoceném grafu, který může být orientovaný nebo neorientovaný. Ohodnocení hran musí být nezáporné, aby byla zaručena správnost algoritmu. Při výskytu záporně ohodnocených hran je vhodné využít pro nalezení nejkratší cesty např. Bellmanův-Fordův algoritmus. Dijkstrův algoritmus je schopný řešit dva typy problémů nejkratších cest, a to jak z počátku do jednoho vrcholu (cíle), tak z počátku do všech ostatních vrcholů. Nejkratší cestou se většinou myslí to první, zatímco to druhé se někdy nazývá strom nejkratších cest.

Nechť je dán orientovaný graf $G = (V, E)$, kde $V$ je množina všech vrcholů grafu $G$ a $E$ je množina všech hran grafu $G$. Každá hrana $(u,v) \in E$ je má nezápornou délku $L(u,v)$. Cílem algoritmu je nalézt nejkratší cestu z počátečního vrcholu $s$ do ostatních vrcholů v grafu. Typicky je však potřeba zjistit nejkratší cestu pouze do jednoho konkrétního vrcholu $t$.  Nejkratší cesta mezi vrcholem $v$ a $w$ je značena $d(v,w)$.

Při postupu algoritmem se vrchol vyskytuje v právě jednom ze tří stavů: *nedosažený*, *dosažený* a *probraný*, kde dosažený vrchol je neprobraný. Vrcholy jsou postupně ohodnoceny číslem $E(v)$ (*estimation*), které udává horní odhad nejkratší cesty z počátku $s$ do vrcholu $v$. Na konci výpočtu je to přímo vzdálenost mezi $s$ a $v$. Ohodnocení vrcholů v jednotlivých stavech je pak následující:

- *Nedosažený* vrchol zatím nebyl ohodnocen, proto $E(v) = \infty$ (nebo též někdy $E(v)$ není definováno)
- *Dosažený* vrchol je vrchol, který byl v průběhu algoritmu prozkoumám a jehož ohodnocení je reálné číslo, nicméně ale ještě nebyl definitivně ohodnocen (a tak nebyl vybrán z prioritní fronty).
- *Probraný* vrchol je takový vrchol, jehož ohodnocení $E(v)$ udává definitivní (nejkratší možnou) vzdálenost z počátku $s$ do vrcholu $v$. 

Pro probrané i dosažené vrcholy $v$ kromě počátku $s$ je navíc dán vrchol $P(v)$, pomocí něhož bude možné dohledat cestu o délce $E(v)$.  [Vrchol?] [\^42] []

Popis algoritmus je dán následujícími kroky \cite{mffDijkstraKucera} [\^10]:
1. pro všechna $v \in V$ označ $E(v) := \infty$, $P(v) := \text{není definováno}$ 
2. počátek $s$ označ jako *dosažený* a $E(s) = 0$
3. dokud existují *objevené* vrcholy, opakuj:
	1. zvol *dosažený* vrchol $u$ takový, kde $E(u) <= E(w)$ pro každý *dosažený* vrchol $w$
	2. označ $u$ jako *probraný*
	3. pro každý vrchol $w$, pro který existuje hrana  $(u,w) \in E$ a platí $E(u) + L(u,w) < E(w)$
		1. pokud je vrchol $w$ *nedosažený*, označ jej jako *dosažený*
		2. označ $E(w) := E(u) + L(u,w)$, $P(w) := u$
4. ukonči algoritmus

Pokud je cílový vrchol $t$ *probraný*, $E(t)$ je délka nejkratší cesty. V opačném případě nejkratší cesta v grafu mezi  $s$  a $t$ neexistuje. Pro nalezení konkrétní cesty včetně všech mezilehlých vrcholů je nutno postupovat od cíle $t$  do počátku $s$ podle hodnot $P(w)$. $P(w)$ udává předchůdce vrcholu $w$ na nejkratší cestě. Při zpětném průchodu se jedná o následovníka. 

Popis algoritmu pro nalezení seznamu vrcholů:
1. vytvoř prázdný seznam $S$
2. nastav $w := t$ a vlož $t$ na začátek seznamu $S$
3. dokud $P(w) \neq \text{není definováno}$:
	1. vlož $P(w)$ na začátek seznamu $S$
	2. $w := P(w)$
4. ukonči algoritmus

Neprázdný seznam $S$ obsahuje výčet probraných vrcholu nejkratší cesty, včetně počátku $s$ a cíle $t$. 

Dijkstra algoritmus běží rychlostí $O(|E|\cdot log|V|)$, když se k výběru vrcholů ze seznamu použije prioritní fronta. [\^40] 

\section{A*}
https://web.archive.org/web/20160322055823/http://ai.stanford.edu/~nilsson/OnlinePubs-Nils/PublishedPapers/astar.pdf

A* (A star) algoritmus je nástavba Dijkstrova algoritmu. Liší se tím, že při procházení vrcholů v grafu je ohodnocení uzlů určeno funkcí $f(v) = g(v) + h(v)$. Jako u Dijkstrova algoritmu, $g(v)$ udává vzdálenost vrcholu $v$ od počátečního uzlu, navíc zde však figuruje heuristická funkce $h(v)$, která udává dolní odhad nejkratší cesty z vrcholu $v$ do cíle. Obecně platí, že čím je heuristika přesnější, tj. že dolní odhad je blízko skutečné vzdálenosti, je algoritmus efektivnější. Je zřejmé, že pokud dolní odhad $h(v) = 0$, tak ekvivalentní zápis ohodnocení vrcholů by byl $f(v) = g(v)$ a jednalo by se tak o standardní Dijkstrův algoritmus. Cílem využití A* je však prohledávání zrychlit, nicméně pro nalezení optimální cesty je nutné zajistit, aby funkce $h(v)$ byla přípustná. Funkce $h(v)$ se nazývá přípustná, právě tehdy když $h(v) \leq d(v, t)$ pro každý vrchol grafu. [\^14]

Přípustná funkce nikdy nepřeceňuje opravdovou vzdálenost od daného vrcholu do cíle, a proto algoritmus A* zaručeně nalezne optimální cestu. [\^7] Intuitivní nástin důkazu: Pokud v grafu existuje unikátní $d(s,t)$, na které neleží vrchol $v$ a $v \neq t$, znamená to, že $d(s,t)$ < $g(v) + h(v)$. Vrchol $v$ tak není probraný, protože algoritmus byl ukončen dříve. Pokud by $d(s,t)$ měla $v$ procházet, muselo by platit $d(s,t) > d(s, v) +d (v,t)$. Vyřešením soustavy nerovnic a $g(v) = d(s,v)$ plyne, že  $h(v) > d(v,t)$, tedy že dolní odhad je větší než skutečná vzdálenost a vrchol $v$ nebyl navštíven. Ovšem platí-li $\forall v \in V, h(v) \leq d(v,t)$, pak dolní odhad je přípustný a každá optimální cesta bude prozkoumána. Formálně dokázáno např. zde [\^15].

*Obrázek, proč musí být přípusné, by se hodil*

Dále je heuristické funkce $h(v)$ monotónní nebo též konzistentní, pokud pro každý pár vrcholů grafu $v$ a $w$ platí:
$$
h(v) \leq d(v,w) + h(w).
$$
Tato vlastnost říká, že odhad délky nejkratší cesty z vrcholu $v$ nesmí být větší než délka hrany $(v,w)$ a odhadu ve vrcholu $w$. Intuitivně to znamená, že odhad nejkratší cesty z $v$ do cíle za podmínky, že cesta prochází $w$, nesmí být menší než bez této podmínky. [\^16] Jedná se tak o trojúhelníková nerovnost, která zajištuje, že jakmile je vrchol probraný, není potřeba jej znova navštěvovat a algoritmus se tak stává efektivnějším. Monotónnost implikuje přípustnost. [ibid]

\subsection{ Základní heuristiky}
Typická přípustná i monotónní heuristická funkce je Eukleidovská metrika, která udává přímou vzdálenost mezi dvěma body (tj. vzdálenost vzdušnou čarou).
$$d(P_1, P_2) = \sqrt{(x_2 - x_1)^2 + (y_2 - y_1)^2}$$

V určitých situacích, kde je pohyb dovolen pouze čtyřmi směry (např. labyrint ve čtvercové síti nebo síť pravoúhlých ulic), je však výhodnější použít Manhattanskou metriku. Ta měří vzdálenost mezi dvěma body jako součet absolutních rozdílu hodnot jejich souřadnic. V porovnání s Eukleidovskou metrikou bude v mřížkové síti Manhattanská metrika přesnější, protože její odhad skutečné vzdálenosti je stejný nebo vyšší, aniž by byla porušena podmínka o přípustnosti. 

Je-li zvolená heuristická funkce přípustná a konzistentní (pro každý probraný uzel platí, že jeho ohodnocení je nejmenší možné), tak A* potřebuje pro nalezení nejkratší cesty objevit méně vrcholů (až 6x méně u 17km cesty v reálné dopravní síti). [\^8]


\subsection{ Heuristika s landmarky}
Zajímavou alternativou ke standardním metrikám výše je využití tzv. landmarků (z angličtiny význačný nebo orientační bod).  Heuristika byla v této formě poprvé popsána zde [\^17] a využívá k určení dolního odhadu trojúhelníkovou nerovnost. Nechť existuje landmark $L$, k němuž existuje nejkratší cesta $d(v,L)$ z vrcholu $v$. Poté z trojúhelníkové nerovnosti plyne, že $d(v,L) +d(v,w) \geq d(L, w)$. Úpravou $d(v,w) \geq d(L, w) - d(v, L)$. Pokud $w = t$, je  dolní odhad $h(v)$  $= |d(L, t) - d(v, L)|$. 

Jinými slovy dolní odhad vzdálenosti nejkratší cesty je absolutní rozdíl vzdálenosti od landmarku k cíli a od aktuálního vrcholu do landmarku. Například je-li to z landmarku do cíle 30km a z vrcholu $v$ do landmarku 50km, nejkratší cesta z $v$ do cíle je nejméně 20km. Pokud je landmarků více, existuje několik takových dolních odhadů. To maximální z nich je nejpřesnější a v konečném důsledku učiní algoritmus efektivnějším. Tato skupina algoritmů se nazývá ALT, jelikož vychází z **A**\* algoritmu a pro výpočet využívá **L**andmarky a **T**rojúhelníkovou nerovnost.

Nevýhoda tohoto přístupu vězí v tom, že je nutno předem spočítat pro každý prvek z množiny vrcholů nejkratší cestu ke všem prvkům v množině landmarků. Je nutné tedy nutné zahrnout fázi předzpracování ("preprocessing"), která může být na velkých grafech několikahodinová. Na druhou stranu nejkratší cesta mezi dvěma body na silniční sítí se jen zřídkakdy mění, proto je věrohodnost předzpracovaných landmarků věrohodná po poměrně dlouhou dobu.

Důležitým aspektem pro tuto heuristiku je výběr landmarků. V původním článku byly navrhnuty metody "random", "planar" a "farthest". [\^18] K nim se pak přidaly metody "avoid" a "maxCover". [\^19] Náhodný výběr je nejrychlejší metoda, ale není vůbec efektivní. Planar využívá znalosti, že silniční síť je geometrický graf, ve kterém délky hran silně korelují se vzdáleností. K nalezení landmarků se graf rozdělí na několik stejně velkých výseku kružnice se středem v centrálním uzlu sítě. Poté je iterativně v každém výseku nalezen takový uzel, který je od centrální nejvzdálenější; je-li blízko hranice s jiným výsekem, je uzel přeskočen. U metody farthest je rovněž snaha nalézt co nejvíce vzdálené uzly. První landmark je zvolen náhodně, zatímco každý další je pak zvolen tak, aby od všech ostatních landmarků byl co nejdále. Několik optimalizací výběru je blíže popsáno zde, jako například opětovné odstranění již vybraných landmarků pro maximalizaci vzájemných vzdáleností.[\^20] 

Metoda "avoid" funguje na té bázi, že se landmark vloží na místo, které aktuálně není dobře pokryto landmarky. U takového místa platí, že dolní odhad je oproti skutečné vzdálenosti příliš malý, a je tam tedy vhodné landmark umístit. Nejsložitější ale zato nejefektivnější je způsob výběru "maxCover", kde je snaha zvolit landmarky tak, aby pro co největší počet vrcholů platilo, že skutečná vzdálenost mezi dvěma vrcholy je právě ta hodnota získána výpočtem vůči alespoň jednomu landmarku. Řešení je vylepšováno iterativně zkoušením několika kandidátských landmarků až je nakonec dosaženo lokální optimum. [\^33]

(alt je rychly [\^25])

\section{Obecné techniky zrychlení Dijkstr}
\subsection{ Oboustranné vyhledávání}
Zásadní technikou pro zrychlení, které se využívá v prakticky všech pokročilejších algoritmech, je oboustranné vyhledávání. [\^57] U standardního Dijkstra algoritmu je graf G prohledáván pouze z počátku do cíle, avšak u oboustranného vyhledávání je tomu tak i z cíle do počátku, akorát v obráceném grafu G. [\^58] Když je jeden vrchol dosažen z obou směrů, k ukončení algoritmu ještě nutně nedochází, ale znamená to, že horní odhad $\mu$ nejkratší cesty je nyní vzdálenost od počátku do $v$ a od cíle do $v$.

Podmínka ukončení: Jakmile je v obou prioritních frontách na prvním místě vrchol, jejichž součet je alespoň tak velký jako $\mu$, byla nalezena nejkratší cesta a algoritmus je ukončena. Podmínku ukončení si lze intuitivně představit tak, že každá nová prozkoumaná hrana nikdy nemůže vytvořit cestu, která by byla menší než $\mu$, protože každý vrchol z fronty musí být spojen s vrcholem, který ještě nebyl dosažen anebo probrán. V obou případech bude $\mu$ překročeno. A tak cesta o délce $\mu$ musí být nejkratší cestou v grafu.


\subsection{ Arc-Flags}
Další metoda vyvinuta Lautherem [\^26] je rovněž náročná na předzpracování, nicméně jednotlivá vyhledávání jsou velmi rychlá. [\^27] Princip Arc-Flags nebo česky "označení hran" vychází z předpokladu, že je možné vyřadit hrany během vyhledávání, pokud hrana nenáleží nejkratší cestě. Tento přístup využívá graf $G$ rozdělený na několik regionů $R_1$… $R_k$, kde každý vrchol náleží právě jednomu regionu. Každá hrana nese $k$ označení. Označení hrany nabývá hodnoty $TRUE$ pro $R_i$, pokud hrana náleží nejkratší cestě alespoň do jednoho uzlu v $R_i$ anebo pokud hrana leží v $R_i$; jinak nabývá $FALSE$. [\^28]

![[Pasted image 20250616223251.png]] 

![[Pasted image 20250619215927.png]] [\^51]

Pro každou hranu, která je na hranici regionu $R_i$, je nutno najít strom nejkratších cest. Pokud je původní graf $G$ orientovaný, je strom nalezen v obráceném grafu $G_{rev}$. Všechny hrany náležící stromu nejkratších cest jsou pak označeny $TRUE$ na pozici $i$. Jakmile jsou všechny hraniční hrany takto označeny, Dijkstrův algoritmus při vyhledávání může vyřadit všechny hrany, které zaručeně nevedou do regionu, ve kterém se nachází cíl. V tomto cílovém regionu je pak postupováno standardním způsobem. Nabízí se však metody několika úrovňového rozdělení, které tuto konečnou fázi algoritmu opět zrychlí za cenu delšího předzpracování. [\^30]Heuristika rozdělení regionů je popsána např. v [\^29].

\subsection{ Highway Hierarchies}
Existuje spousta dalších technik pro urychlení základního Dijkstra algoritmu, které se navíc dají mezi sebou různě kombinovat. Za zmínku ještě stojí skupina algoritmů, které člení graf do více úrovní a vytváří jakousi hierarchii. [\^31] 

Prvním z nich je "Highway hierarchies", který využívá intuitivní myšlenky, že při cestování na dlouhé vzdálenosti jsou dálnice mnohem důležitější než silnice II. a III. tříd nebo místní komunikace. Tyto silnice jsou důležité pouze v malé oblasti kolem počátku a cíle cesty. Tato metoda tak převádí graf do několika úrovní, které volně pasují na výše zmíněné typy silnic, nicméně úrovní v grafu může být o mnoho více a nevyžadují manuální uživatelský vstup.

Formálněji, graf je opět rozdělen na několik lokálních regionů nebo sousedství. Není-li počátek a cíl ve stejném lokálním regionu, je možné najít takové hrany, které se nacházejí na nejkratší cestě. Tyto hrany jsou tzv. "highway" hrany. Z vrcholů těchto hran se vytvoří maximální vrcholově indukovaný podgraf se stupněm každého vrcholu alespoň dva. [\^43] [\^44]  V podgrafu se pak zruší každý vnitřní vrchol $v$ (vrchol se stupněm právě dva), který sousedí s vrcholy $u$ a $w$, společně s hranami $v$ incidujícími. Pro zachování struktury jsou tyto hrany nahrazeny jednou novou hranou, jejíž délka je rovna zrušeným incidujícím hranám.  Tímto zkrácením (kontrakcí) se vytvoří síť dálnic.

![[Pasted image 20250618212053.png]]
[\^45]

![[Pasted image 20250617222517.png]]
[\^36]

~~V návaznosti je pak možné snížit počet počet vrcholů tím, že se méně důležité uzly (např. nájezdy na dálnici) vypustí.~~ 

\subsection{ Contraction Hierachies }
Další metoda navazuje právě na ten poslední krok minulého algoritmu, tedy na zkracování (kontrakci) nepotřebných vrcholů a nahrazování hran zkratkami; po zkracování nese pojmenována:  "Contraction hierarchies". [\^34] [\^35] [\^39]  U něj je navíc specifické, že vytváří hierarchii všech *vrcholů* v grafu podle pořadí, v jakém byly zkráceny. Je-li pořadí efektivně zvoleno, budou dříve zkracované vrcholy méně důležité (jako např. křižovatka se slepou ulicí), zatímco ty později zkracované budou hrát větší dopravní úlohu. Při postupu algoritmem v dopředném směru jsou skenovány pouze hrany vedoucí do vrcholů vyššího řádu, zatímco v zpětném směru jsou zkoumány hrany, které vycházejí z výše postavených vrcholů.
 
Zkracování probíhá následovně: když je vrchol $v$ zkracován, jsou prověřeny nejkratší cesty mezi všemi páry sousedů $v$. Pokud nejkratší cesta mezi dvěma sousedy $u$ a $w$ vede přes $v$, je vytvořena nová hrana, tzv. zkratka. Není-li $v$ na nejkratší cestě (existuje tzv. *witness*), hrany jsou zrušeny bez náhrady. Například na obr. byly při zkracování vrcholu $v$ celkově zrušeny 4 hrany a tři byly přidány. Takhle jsou postupně zkráceny všechny hrany grafu. Ideální je při zkracování postupovat od těch hran, kde rozdíl zrušených a nově vytvořených hran je co největší, aby graf zbytečně nerostl na velikosti. Některé heuristiky zrychlení tohoto procesu jsou popsány např. zde. [\^37] [\^38]

![[Pasted image 20250618162852.png]]
Napr r a w by mely lepsi zruseni hranu, nepridaly by se dalsi nove hrany

Pro vyhledání nejkratší cesty u orientované grafu G pokračuje algoritmus tak, že postupně v popředném směru od počátku skenuje podgraf G, který obsahuje orientované hrany mezi $v$ a $w$, pokud byl $v$ zkrácen *před* $w$. Zároveň zpětný algoritmus skenuje orientované hrany mezi $v$ a $w$, pokud byl $v$ zkrácen *po* $w$, a to v obráceném grafu G. Jinými slovy, oba algoritmy "stoupají" v hierarchii vrcholů, dokud není splněna podmínka ukončení viz. oboustranné vyhledávaní. [\^49] 

Pro získání přesné cesty je pak nutné rekurzivně rozbalit jednotlivé zkratky, kde každá z nich postupně obsahují nejpozději zkrácený vrchol.

~~Na podobné bázi je založen například i algoritmus Reach, kde vrcholy, které se vyskytují na relativně dlouhých nejkratších cestách, jsou důležitější než vrcholy na relativně krátkých nejkratších cestách. [\^32]~~

Přehled dalších algoritmů jako např. SHARC nebo Hub Labeling je zde [\^47] [\^50]

